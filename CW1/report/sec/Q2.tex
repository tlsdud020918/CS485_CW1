\section{Incremental PCA}
\label{sec:intro}

Please follow the steps outlined below when submitting your manuscript to the IEEE Computer Society Press.
This style guide now has several important modifications (for example, you are no longer warned against the use of sticky tape to attach your artwork to the paper), so all authors should read this new version.

%-------------------------------------------------------------------------
\subsection{Important parameter in implementation: $d_3$}
To implement incremental PCA, we utilized the algorithm from the "Online Learning" slides presented in class. Here, a key parameter is $d_2$ and $d_3)$. When new data arrives in incremental PCA, computing the eigenspace model for this subset requires $O(\min(D, N')^3)$ time, where $N'$ is the number of data points in the subset. Additionally, merging this new eigenspace model with the existing data takes $O((d_1 + d_2 + 1)^3)$ time, where $d_1$ is equal to the previously computed eigenspace model's $d_3$ value. Therefore, to enhance time efficiency in incremental PCA, it is essential to keep $d_3$ small, although this results in a time-accuracy tradeoff by dropping less-significant eigenvector information, which is represented by our experiment result \cref{fig:q2-fig5}

\subsection{Comparison with other PCA}

All manuscripts must be in English.